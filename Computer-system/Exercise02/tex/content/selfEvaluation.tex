\section{Đánh giá}
\subsection{Bảng tự đánh giá các yêu cầu đã hoàn thành}

\begin{center}
\begin{table}[H]
    \centering
    \caption{Bảng tự đánh giá bài 1}
    \renewcommand{\arraystretch}{1.4}
    \begin{tabular}{|l|p{\dimexpr0.6\linewidth-2\tabcolsep}|c|}
    \hline
    \textbf{STT} & \textbf{Yêu cầu}            & \textbf{Mức độ hoàn thành} \\ \hline
    1            &  Viết chương trình nhập vào số chấm động. Hãy xuất ra biểu diễn nhị phân từng thành phần 
    (dấu, phần mũ, phần trị) của số chấm động vừa nhập       & 100\%          \\ \hline
                & \textbf{Tổng cộng}               &\textbf{100\%}           \\ \hline
    \end{tabular}
    \label{tab:mytable}
\end{table}

\begin{table}[H]
  \centering
  \caption{Bảng tự đánh giá bài 2}
  \renewcommand{\arraystretch}{1.4}
  \begin{tabular}{|l|p{\dimexpr0.6\linewidth-2\tabcolsep}|c|}
  \hline
  \textbf{STT} & \textbf{Yêu cầu}            & \textbf{Mức độ hoàn thành} \\ \hline
  1            &  Viết chương trình nhập vào số chấm động. Hãy xuất ra biểu diễn nhị phân từng thành phần 
  (dấu, phần mũ, phần trị) của số chấm động vừa nhập       & 100\%          \\ \hline
              & \textbf{Tổng cộng}               &\textbf{100\%}           \\ \hline
  \end{tabular}
  \label{tab:mytable}
\end{table}

    \begin{table}[H]
        \centering
        \caption{Bảng tự đánh giá bài 3}
        \renewcommand{\arraystretch}{1.4}
        \begin{tabular}{|l|p{\dimexpr0.6\linewidth-2\tabcolsep}|c|}
        \hline
        \textbf{STT} & \textbf{Yêu cầu}            & \textbf{Mức độ hoàn thành} \\ \hline
        1            &  Biểu diễn nhị phân số 1.3E+20   & 100\%          \\ \hline
        2            &  Biểu diễn nhị phân số float nhỏ nhất lớn hơn 0     & 100\%          \\ \hline
        3            &  Những trường hợp tạo ra các số đặc biệt      & 100\%          \\ \hline
                    & \textbf{Tổng cộng}               &\textbf{100\%}           \\ \hline
      \end{tabular}
        \label{tab:mytable2}
    \end{table}

    \begin{table}[H]
      \centering
      \caption{Bảng tự đánh giá bài 4}
      \renewcommand{\arraystretch}{1.4}
      \begin{tabular}{|l|p{\dimexpr0.6\linewidth-2\tabcolsep}|c|}
      \hline
      \textbf{STT} & \textbf{Yêu cầu}            & \textbf{Mức độ hoàn thành} \\ \hline
      1            &  Khảo sát chuyển đổi float -> int -> float     & 100\%         \\ \hline
      2            &  Khảo sát chuyển đổi int -> float -> int     & 100\%          \\ \hline
      3            &  Khảo sát tính kết hợp của phép cộng      & 100\%          \\ \hline
      4            &  Khảo sát i = (int) (3.14159 * f)      & 100\%          \\ \hline
      5            &  Khảo sát f = f + (float) i;       & 100\%          \\ \hline
      6            &  Khảo sát if (i == (int)((float) i)) { printf(“true”); }       & 100\%          \\ \hline
      7            &  Khảo sát if (i == (int)((double) i)) { printf(“true”); }      & 100\%          \\ \hline
      8            &  Khảo sát if (f == (float)((int) f)) { printf(“true”); }       & 100\%          \\ \hline
      9            &  Khảo sát if (f == (double)((int) f)) { printf(“true”); }       & 100\%          \\ \hline

                  & \textbf{Tổng cộng}               &\textbf{100\%}           \\ \hline
    \end{tabular}
      \label{tab:mytable2}
  \end{table}
\end{center}

\subsection{Đánh giá tổng thể mức độ hoàn thành của bài nộp}

Bài nộp đã hoàn thành đầy đủ các yêu cầu đề ra trong bài tập. Tất cả các yêu cầu đều đã được cài đặt và kiểm thử thành công. Tổng thể, bài nộp đã hoàn thành 100\% các yêu cầu đề ra.
