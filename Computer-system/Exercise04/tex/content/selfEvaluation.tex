\section{Đánh giá}
\subsection{Bảng tự đánh giá các yêu cầu đã hoàn thành}

\begin{center}
  \begin{table}[H]
    \centering
    \caption{Bảng tự đánh giá bài 1}
    \renewcommand{\arraystretch}{1.4}
    \begin{tabular}{|l|p{\dimexpr0.6\linewidth-2\tabcolsep}|c|}
      \hline
      \textbf{STT} & \textbf{Yêu cầu}                                                              & \textbf{Mức độ hoàn thành} \\ \hline
      1            & Viết chương trình nhập số nguyên n. Kiểm tra n có là số nguyên tố hay không ? & 100\%                      \\ \hline
                   & \textbf{Tổng cộng}                                                            & \textbf{100\%}             \\ \hline
    \end{tabular}
    \label{tab:mytable}
  \end{table}

  \begin{table}[H]
    \centering
    \caption{Bảng tự đánh giá bài 2}
    \renewcommand{\arraystretch}{1.4}
    \begin{tabular}{|l|p{\dimexpr0.6\linewidth-2\tabcolsep}|c|}
      \hline
      \textbf{STT} & \textbf{Yêu cầu}                                                                      & \textbf{Mức độ hoàn thành} \\ \hline
      1            & Viết chương trình nhập số nguyên n. Kiểm tra n có là số nguyên hoàn thiện hay không ? & 100\%                      \\ \hline
                   & \textbf{Tổng cộng}                                                                    & \textbf{100\%}             \\ \hline
    \end{tabular}
    \label{tab:mytable}
  \end{table}


  \begin{table}[H]
    \centering
    \caption{Bảng tự đánh giá bài 3}
    \renewcommand{\arraystretch}{1.4}
    \begin{tabular}{|l|p{\dimexpr0.6\linewidth-2\tabcolsep}|c|}
      \hline
      \textbf{STT} & \textbf{Yêu cầu}                                                                     & \textbf{Mức độ hoàn thành} \\ \hline
      1            & Viết chương trình nhập vào số nguyên n. Kiểm tra n có là số chính phương hay không ? & 100\%                      \\ \hline
                   & \textbf{Tổng cộng}                                                                   & \textbf{100\%}             \\ \hline
    \end{tabular}
    \label{tab:mytable}
  \end{table}


  \begin{table}[H]
    \centering
    \caption{Bảng tự đánh giá bài 4}
    \renewcommand{\arraystretch}{1.4}
    \begin{tabular}{|l|p{\dimexpr0.6\linewidth-2\tabcolsep}|c|}
      \hline
      \textbf{STT} & \textbf{Yêu cầu}                                                             & \textbf{Mức độ hoàn thành} \\ \hline
      1            & Viết chương trình nhập số nguyên n. Kiểm tra n có là số đối xưng hay không ? & 100\%                      \\ \hline
                   & \textbf{Tổng cộng}                                                           & \textbf{100\%}             \\ \hline
    \end{tabular}
    \label{tab:mytable}
  \end{table}



  \begin{table}[H]
    \centering
    \caption{Bảng tự đánh giá bài 4}
    \renewcommand{\arraystretch}{1.4}
    \begin{tabular}{|l|p{\dimexpr0.6\linewidth-2\tabcolsep}|c|}
      \hline
      \textbf{STT} & \textbf{Yêu cầu}                                         & \textbf{Mức độ hoàn thành} \\ \hline
      1            & Nhập mảng 1 chiều n phần tử số nguyên                    & 100\%                      \\ \hline
      2            & Xuất mảng                                                & 100\%                      \\ \hline
      3            & Liệt kê các số nguyên tố                                 & 100\%                      \\ \hline
      4            & Tìm giá trị lớn nhất trong mảng                          & 100\%                      \\ \hline
      5            & Tính trung bình mảng                                     & 100\%                      \\ \hline

                   & \textbf{Tổng cộng}                                       & \textbf{100\%}             \\ \hline
    \end{tabular}
    \label{tab:mytable2}
  \end{table}
\end{center}

\subsection{Đánh giá tổng thể mức độ hoàn thành của bài nộp}

Bài nộp đã hoàn thành đầy đủ các yêu cầu đề ra trong bài tập. Tất cả các yêu cầu đều đã được cài đặt bằng kỹ thuật hàm và kiểm thử thành công. Tổng thể, bài nộp đã hoàn thành 100\% các yêu cầu đề ra.
