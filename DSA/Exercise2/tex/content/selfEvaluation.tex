\section{Self - Evaluation}
In this exercise, I meticulously implemented Hash Table from scratch in C++. My evaluation encompassed several critical aspects.
\begin{itemize}
    \item I verified the correctness of both implementations, ensuring that they adhered to the fundamental properties of Hash Table.
    \item I optimized the code for performance, using efficient algorithms and data structures. I also minimized the number of comparisons and avoided redundant operations.
    \item I paid close attention to memory usage, avoiding unnecessary wastage.
    \item I prioritized code readability and maintainability, using descriptive names and adding explanatory comments. Lastly, I thoroughly tested edge cases and implemented robust error handling. The code is modular, well-commented, and meets the specified criteria.
\end{itemize}

\begin{center}
    \begin{table}[H]
        \centering
        \caption{Self - Evaluation about my Exercise}
        \renewcommand{\arraystretch}{1.4}
        \begin{tabular}{|l|p{\dimexpr0.6\linewidth-2\tabcolsep}|c|}
            \hline
            \textbf{No.} & \textbf{Details}           & \textbf{Score} \\ \hline
            1            & Linear Probing             & 100\%          \\ \hline
            2            & Quadratic Probing          & 100\%          \\ \hline
            3            & Chaining using Linked List & 100\%          \\ \hline
            4            & Chaining using AVL Tree    & 100\%          \\ \hline
            5            & Double Hashing             & 100\%          \\ \hline
            6            & Experiments                & 100\%          \\ \hline
            7            & Report                     & 100\%          \\ \hline
                         & \textbf{Total}             & \textbf{100\%} \\ \hline
        \end{tabular}
        \label{tab:mytable}
    \end{table}

\end{center}

