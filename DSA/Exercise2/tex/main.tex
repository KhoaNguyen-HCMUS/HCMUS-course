\documentclass[12pt]{article}
\usepackage[table,xcdraw]{xcolor}
\usepackage{multirow}
\usepackage{pdflscape}
\usepackage{amsmath}
\usepackage{amsfonts}
\usepackage{float}
\usepackage{fancyhdr}
\usepackage{graphicx}
\usepackage{hyperref}
\usepackage{indentfirst}
\usepackage{url}
\usepackage{subfig}
\usepackage[top=.75in, left=.75in, right=.75in, bottom=1in]{geometry}
\usepackage{gensymb}
\hypersetup{
    colorlinks=true,
    linkcolor=blue,
    filecolor=magenta,      
    urlcolor=cyan,
    pdftitle={Overleaf Example},
    pdfpagemode=FullScreen,
    }
%\usepackage[utf8]{vietnam}

% For algorithm
\usepackage{algorithm}
\usepackage{algpseudocode}

% ============ CODE ============
\usepackage{listings} 
\usepackage{xcolor}
\definecolor{codegreen}{rgb}{0,0.6,0}
\definecolor{codegray}{rgb}{0.5,0.5,0.5}
\definecolor{codepurple}{rgb}{0.58,0,0.82}
\definecolor{backcolour}{rgb}{0.95,0.95,0.92}

% Styling for the code.
\lstdefinestyle{mystyle}{
    backgroundcolor=\color{backcolour},   
    commentstyle=\color{codegreen},
    keywordstyle=\color{magenta},
    numberstyle=\tiny\color{codegray},
    stringstyle=\color{codepurple},
    basicstyle=\ttfamily\footnotesize,
    breakatwhitespace=false,         
    breaklines=true,                 
    captionpos=b,                    
    keepspaces=true,                 
    numbers=left,                    
    numbersep=5pt,                  
    showspaces=false,                
    showstringspaces=false,
    showtabs=false,                  
    tabsize=2
}
\lstset{style=mystyle}

% Disable indentation on new paragraphs
\setlength{\parindent}{0pt}

% Optional: graphic path
% \graphicspath{PATH_TO_GRAPHIC_FOLDER}

% To use Times font family, uncomment this row
% \usepackage{mathptmx}

% To use roman section / subsection, uncomment these rows
% \renewcommand{\thesection}{\Roman{section}}
% \renewcommand{\thesubsection}{\thesection.\Roman{subsection}}

% Define course name, report name and report title.
\newcommand{\coursename}{Data Structures and Algorithms}
\newcommand{\reportname}{Exercise 2: Implementing Hash Table from scratch}
\newcommand{\reporttitle}{Report}

\newcommand{\studentname}{Nguyen Le Ho Anh Khoa - 23127211}
\newcommand{\teachername}{Bui Duy Dang \\ Truong Tan Khoa \\ Nguyen Thanh Tinh}

% ============ HEADER AND FOOTER ============
% Header length
\setlength{\headheight}{29.43912pt}

% Footer page number would be on the lower-right corner
\pagestyle{fancy}
\fancyfoot{}
\fancyfoot[R]{Page \thepage}

\lhead{\reportname}
\rhead{VNUHCM-US\\
\coursename
}
\lfoot{}

%Footer page number for landscape
\fancypagestyle{lscape}{
\fancyhf{}
\fancyfoot[R]{Page \thepage}
\renewcommand{\headrulewidth}{0pt} 
  \renewcommand{\footrulewidth}{0pt}
}



% ============ DOCUMENT ============
\begin{document}
\begin{titlepage}
\newcommand{\HRule}{\rule{\linewidth}{0.5mm}}
\centering

\textsc{\LARGE vietnam national university of \\ho chi minh city}\\[0.8cm]
\textsc{\Large university of science}\\[0.4cm]
\textsc{\large faculty of information technology}\\[0.4cm]

\HRule \\[0.4cm]
{ 
\Large{\bfseries{\reporttitle}}\\[0.4cm]
\huge{\bfseries{\reportname}}
}\\[0.4cm]
\HRule \\[0.4cm]

\textbf{\large Course name: \coursename}\\[0.4cm]
\textbf{\large CSC10004\textunderscore23CLC09} \\ [0.7cm]
\begin{minipage}[t]{0.4\textwidth}
\begin{flushleft} \large
\emph{Students:}\\
\studentname
\end{flushleft}
\end{minipage}
~
\begin{minipage}[t]{0.4\textwidth}
\begin{flushright} \large
\emph{Teacher:} \\
\teachername
\end{flushright}
\end{minipage}\\[0.7cm]

{\large \today}\\[1cm]

\includegraphics[scale=1.1]{img/hcmus-logo.png}\\[0cm] 

\vfill
\end{titlepage}
	
\tableofcontents
\pagebreak
\section{Student Information}
Class: 23CLC09 \\
Student ID: 23127211 \\
Full name: Nguyen Le Ho Anh Khoa
\section{How I implemented the requirements}

To hash a string, you used polynomial rolling hash function as the hints in requirements. The formula is:
\begin{equation}
	\text{hash}(s) = \left( \sum_{i = 0}^{n - 1} (s[i] \times p^i) \right) \mod m
\end{equation}
where:
\begin{itemize}
	\item $s$: The key as a string of length $n$.
	\item $s[i]$: ASCII code of the character at position $i$ from $s$.
	\item $p = 31$.
	\item $m = 10^9 + 9$.
\end{itemize}
To hash the keys, I used the following hash functions:
\begin{itemize}
	\item Linear Probing: $h(k, i) = (h'(k) + i) \mod capacity$. $h'(k)$ is the hash value of key $k$.
	\item Double Hashing: $h(k, i) = (h_1(k) + i \times h_2(k)) \mod capacity$. $h_1(k)$ is the first hash value of key $k$, and $h_2(k)$ is the second hash value of key $k$. $h_2(k) = 1 + h(k) \mod (capacity -1) $.
	\item Quadratic Probing: $h(k, i) = (h'(k) + i^2) \mod capacity$.  $h'(k)$ is the hash value of key $k$.
	\item Chaining using Linked List: $h(k, i) = (h'(k) + i) \mod capacity$. $h'(k)$ is the hash value of key $k$.  Each cell of the hash table is a linked list. When a collision occurs, the new element is inserted at the end of the linked list.
	\item Chaining using AVL Tree: $h(k, i) = (h'(k) + i) \mod capacity$. $h'(k)$ is the hash value of key $k$. Each cell of the hash table is an AVL tree. When a collision occurs, the new element is inserted into the AVL tree.
\end{itemize}

\textbf{Quadratic probing} and \textbf{Double hashing} frequently encounter collisions when the hash table size is small or when keys hash to the same index. This can quickly lead to revisiting the same indices, especially if the capacity is a prime number.\\

To address this issue, a rehash function needs to be implemented to increase the size of the hash table and redistribute the elements. The rehash function creates a new hash table with double size and transfers all elements from the old hash table to the new one. This helps to minimize collisions and improve the performance of the hash table.




\pagebreak
\section{Detailed Experiments}
\label{sec:detailedExperiments}

\subsection{Linear Probing}


\subsubsection{Add key:}
The order insert key to the Hash Table by Linear Probing is as follows: [one; 1], [two; 2], [three; 3], [four; 4], [five; 5], [one; 111], [two; 222]. The size of the Hash Table is 5. The Hash Table is shown in the figure below:

\begin{figure}[H]
	\centering
	\subfloat[\centering Add new]{{\includegraphics[width=7cm]{img/Linear/addnew.PNG} }}%
	\qquad
	\subfloat[\centering Update]{{\includegraphics[width=7cm]{img/Linear/updateValue.PNG} }}%
	\caption{Add new by key by Linear Probing}%
\end{figure}

\subsubsection{Search key (Use the Hash Table added as above 3.1.1):}
\begin{figure}[H]
	\centering
	\subfloat[\centering Found]{{\includegraphics[width=7cm]{img/Linear/Found.PNG} }}%
	\qquad
	\subfloat[\centering Not Found]{{\includegraphics[width=7cm]{img/Linear/Not found.PNG} }}%
	\caption{Search value by key by Linear Probing (size=5)}%
\end{figure}

\subsubsection{Remove key (Use the Hash Table added as above 3.1.1):}
\begin{figure}[H]
	\centering
	\subfloat[\centering Removed]{{\includegraphics[width=7cm]{img/Linear/removed.PNG} }}%
	\qquad
	\subfloat[\centering Not Found]{{\includegraphics[width=7cm]{img/Linear/removeNotFound.PNG} }}%
	\caption{Remove value by key by Linear Probing}%
\end{figure}

\subsubsection{Experiments Linear Probing and Linear Search Algorithms}
\begin{figure}[H]
	\centering
	\includegraphics[width=0.7\linewidth]{img/Linear/Compare.PNG}
	\caption{Compare time Linear Probing Search and Linear Search Algorithms}
\end{figure}

\begin{itemize}
	\item The time complexity of Linear Probing Search is O(n)
	\item The time complexity of Linear Search Algorithms is O(n).
\end{itemize}
But in the most cases, using Linear Probing Search in Hash Table is faster than using Linear Search Algorithms in normal vector.

\pagebreak
\subsection{Quadratic Probing}


\subsubsection{Add key:}
The order insert key to the Hash Table by Quadratic Probing is as follows: [one; 1], [two; 2], [three; 3], [four; 4], [five; 5], [one; 111], [two; 222]. The size of the Hash Table is 5. But after added the key [three; 3], the size of the Hash Table is 10 by rehashing. However, this condition can be met even if there are still empty slots available, due to the nature of quadratic probing. The Hash Table is shown in the figure below:

\begin{figure}[H]
	\centering
	\subfloat[\centering Add new]{{\includegraphics[width=7cm]{img/Quadratic/addnew.PNG} }}%
	\qquad
	\subfloat[\centering Update]{{\includegraphics[width=7cm]{img/Quadratic/update.PNG}}}%
	\caption{Add new key by Quadratic Probing (size=5)}%
\end{figure}

\subsubsection{Search key (Use the Hash Table added as above 3.2.1):}
\begin{figure}[H]
	\centering
	\subfloat[\centering Found]{{\includegraphics[width=7cm]{img/Quadratic/Found.PNG} }}%
	\qquad
	\subfloat[\centering Not Found]{{\includegraphics[width=7cm]{img/Quadratic/notFound.PNG} }}%
	\caption{Search value by key by Quadratic Probing (size=5)}%
\end{figure}

\subsubsection{Remove key (Use the Hash Table added as above 3.2.1):}
\begin{figure}[H]
	\centering
	\subfloat[\centering Removed]{{\includegraphics[width=7cm]{img/Quadratic/removed.PNG} }}%
	\qquad
	\subfloat[\centering Not Found]{{\includegraphics[width=7cm]{img/Quadratic/removeNotFound.PNG} }}%
	\caption{Remove value by key by Quadratic Probing}%
\end{figure}

\subsubsection{Experiments Quadratic Probing and Linear Search Algorithms}
\begin{figure}[H]
	\centering
	\includegraphics[width=0.7\linewidth]{img/Quadratic/compare.PNG}
	\caption{Compare time Quadratic Probing and Linear Search Algorithms}
\end{figure}

\begin{itemize}
	\item The time complexity of Quadratic Probing Search is O(n)
	\item The time complexity of Linear Search Algorithms is O(n).
\end{itemize}
But in the most cases, using Quadratic Probing Search in Hash Table is faster than using Linear Search Algorithms in normal vector.\\

\pagebreak
\subsection{Chaining using Linked List}

\subsubsection{Add key:}
The order insert key to the Hash Table by Chaining using Linked List is as follows: [one; 1], [two; 2], [three; 3], [four; 4], [five; 5], [one; 111], [two; 222]. The size of the Hash Table is 5. The Hash Table is shown in the figure below:

\begin{figure}[H]
	\centering
	\subfloat[\centering Add new]{{\includegraphics[width=7cm]{img/ChainingLinkedList/addnew.PNG} }}%
	\qquad
	\subfloat[\centering Update]{{\includegraphics[width=7cm]{img/ChainingLinkedList/update.PNG} }}%
	\caption{Add new by key by Chaining using Linked List}%
\end{figure}

\subsubsection{Search key (Use the Hash Table added as above 3.3.1):}
\begin{figure}[H]
	\centering
	\subfloat[\centering Found]{{\includegraphics[width=7cm]{img/ChainingLinkedList/found.PNG} }}%
	\qquad
	\subfloat[\centering Not Found]{{\includegraphics[width=7cm]{img/ChainingLinkedList/notfound.PNG} }}%
	\caption{Search value by key by Chaining using Linked List}%
\end{figure}

\subsubsection{Remove key (Use the Hash Table added as above 3.3.1):}
\begin{figure}[H]
	\centering
	\subfloat[\centering Removed]{{\includegraphics[width=7cm]{img/ChainingLinkedList/found.PNG} }}%
	\qquad
	\subfloat[\centering Not Found]{{\includegraphics[width=7cm]{img/ChainingLinkedList/removeNotFound.PNG} }}%
	\caption{Remove value by key by Chaining Using Linked List}%
\end{figure}

\subsubsection{Experiments Chaining Using Linked List and Linear Search Algorithms}
\begin{figure}[H]
	\centering
	\includegraphics[width=0.7\linewidth]{img/ChainingLinkedList/compare.PNG}
	\caption{Compare time Chaining using Linked List Search and Linear Search Algorithms}
\end{figure}

\begin{itemize}
	\item The time complexity of Chaining using Linked List Search is O(n)
	\item The time complexity of Linear Search Algorithms is O(n).
\end{itemize}
But in the most cases, using Chaining using Linked List Search in Hash Table is faster than using Linear Search Algorithms in normal vector.

\pagebreak
\subsection{Chaining using AVL Tree}
\subsubsection{Add key:}
The order insert key to the Hash Table by Chaining using AVL Tree is as follows: [one; 1], [two; 2], [three; 3], [four; 4], [five; 5], [one; 111], [two; 222]. The size of the Hash Table is 5. The Hash Table is shown in the figure below:

\begin{figure}[H]
	\centering
	\subfloat[\centering Add new]{{\includegraphics[width=7cm]{img/ChainingAVL/addnew.PNG} }}%
	\qquad
	\subfloat[\centering Update]{{\includegraphics[width=7cm]{img/ChainingAVL/update.PNG} }}%
	\caption{Add new by key by Chaining using AVL Tree}%
\end{figure}

\subsubsection{Search key (Use the Hash Table added as above 3.4.1):}
\begin{figure}[H]
	\centering
	\subfloat[\centering Found]{{\includegraphics[width=7cm]{img/ChainingAVL/found.PNG} }}%
	\qquad
	\subfloat[\centering Not Found]{{\includegraphics[width=7cm]{img/ChainingAVL/notfound.PNG} }}%
	\caption{Search value by key by CChaining Using AVL Tree}%
\end{figure}

\subsubsection{Remove key (Use the Hash Table added as above 3.4.1):}
\begin{figure}[H]
	\centering
	\subfloat[\centering Removed]{{\includegraphics[width=7cm]{img/ChainingAVL/removed.PNG} }}%
	\qquad
	\subfloat[\centering Not Found]{{\includegraphics[width=7cm]{img/ChainingAVL/removeNotFound.PNG} }}%
	\caption{Remove value by key by Chaining Using AVL Tree}%
\end{figure}

\subsubsection{Experiments Chaining using AVL Tree and Linear Search Algorithms}
\begin{figure}[H]
	\centering
	\includegraphics[width=0.7\linewidth]{img/ChainingAVL/compare.PNG}
	\caption{Compare time Chaining using AVL Tree Search and Linear Search Algorithms}
\end{figure}

\begin{itemize}
	\item The time complexity of Chaining using AVL Tree Search is O(logn)
	\item The time complexity of Linear Search Algorithms is O(n).
\end{itemize}
So in the most cases, using Chaining using AVL Tree Search in Hash Table is faster than using Linear Search Algorithms in normal vector.


\pagebreak
\subsection{Double Hashing}

\subsubsection{Add key:}
The order insert key to the Hash Table size = 5 by Double Hashing is as follows: [one; 1], [two; 2], [three; 3], [four; 4], [five; 5], [one; 111], [two; 222].The Hash Table is shown in the figure below:

\begin{figure}[H]
	\centering
	\subfloat[\centering Add new]{{\includegraphics[width=7cm]{img/DoubleHash/addnew.PNG} }}%
	\qquad
	\subfloat[\centering Update]{{\includegraphics[width=7cm]{img/DoubleHash/update.PNG}}}%
	\caption{Add new key by Quadratic Probing (size=5)}%
\end{figure}

\subsubsection{Search key (Use the Hash Table added as above 3.5.1):}
\begin{figure}[H]
	\centering
	\subfloat[\centering Found]{{\includegraphics[width=7cm]{img/DoubleHash/found.PNG} }}%
	\qquad
	\subfloat[\centering Not Found]{{\includegraphics[width=7cm]{img/DoubleHash/notfound.PNG} }}%
	\caption{Search value by key by Quadratic Probing (size=5)}%
\end{figure}

\subsubsection{Remove key (Use the Hash Table added as above 3.5.1):}
\begin{figure}[H]
	\centering
	\subfloat[\centering Removed]{{\includegraphics[width=7cm]{img/DoubleHash/removed.PNG} }}%
	\qquad
	\subfloat[\centering Not Found]{{\includegraphics[width=7cm]{img/DoubleHash/removeNotFound.PNG} }}%
	\caption{Remove value by key by Double Hashing}%
\end{figure}

\subsubsection{Experiments Double Hashing and Linear Search Algorithms}
\begin{figure}[H]
	\centering
	\includegraphics[width=0.7\linewidth]{img/DoubleHash/compare.PNG}
	\caption{Compare time Double Hashing Search and Linear Search Algorithms}
\end{figure}

\begin{itemize}
	\item The time complexity of Double Hashing Search is O(n)
	\item The time complexity of Linear Search Algorithms is O(n).
\end{itemize}
But in the most cases, using Double Hashing Search in Hash Table is faster than using Linear Search Algorithms in normal vector.\\

When run with large dataset, Hash Table by Double Hashing need to rehash because the condition \textbf{ if (i == capacity)} is used to determine if the table is full. However, this condition can be met even if there are still empty slots available, due to the nature of quadratic probing.

\section{Self - Evaluation}
In this exercise, I meticulously implemented Hash Table from scratch in C++. My evaluation encompassed several critical aspects.
\begin{itemize}
    \item I verified the correctness of both implementations, ensuring that they adhered to the fundamental properties of Hash Table.
    \item I optimized the code for performance, using efficient algorithms and data structures. I also minimized the number of comparisons and avoided redundant operations.
    \item I paid close attention to memory usage, avoiding unnecessary wastage.
    \item I prioritized code readability and maintainability, using descriptive names and adding explanatory comments. Lastly, I thoroughly tested edge cases and implemented robust error handling. The code is modular, well-commented, and meets the specified criteria.
\end{itemize}

\begin{center}
    \begin{table}[H]
        \centering
        \caption{Self - Evaluation about my Exercise}
        \renewcommand{\arraystretch}{1.4}
        \begin{tabular}{|l|p{\dimexpr0.6\linewidth-2\tabcolsep}|c|}
            \hline
            \textbf{No.} & \textbf{Details}           & \textbf{Score} \\ \hline
            1            & Linear Probing             & 100\%          \\ \hline
            2            & Quadratic Probing          & 100\%          \\ \hline
            3            & Chaining using Linked List & 100\%          \\ \hline
            4            & Chaining using AVL Tree    & 100\%          \\ \hline
            5            & Double Hashing             & 100\%          \\ \hline
            6            & Experiments                & 100\%          \\ \hline
            7            & Report                     & 100\%          \\ \hline
                         & \textbf{Total}             & \textbf{100\%} \\ \hline
        \end{tabular}
        \label{tab:mytable}
    \end{table}

\end{center}


\section{Exercise Feedback}

\subsection{What have I learned}
\begin{itemize}
    \item In the past, my approach was sequential programming, but this task has broadened my knowledge to include the fundamentals of object-oriented programming.
    \item I’ve acquired skills in creating reports with \LaTeX.
    \item I’ve employed Github as a vault for my source code and reports, all of which are securely stored in \href{https://github.com/KhoaNguyen-HCMUS/HCMUS-course}{my personal repository}.
\end{itemize}
\subsection{What was my difficult}
\begin{itemize}
    \item At the outset, my journey with programming was fraught with difficulties due to my unfamiliarity with object-oriented programming. However, my comprehension has been greatly enhanced after delving into a plethora of resources available on the internet. \cite{OOP_in_C++}
    \item I encountered some difficulties when writing function about AVL tree. However, after a period of debugging and contemplation, I successfully resolved the issues. \cite{AVL_Tree}
    \item I had a hard time understanding the concept of hash tables, but after a period of research and experimentation, I was able to grasp the fundamentals.
    \item Measuring the time complexity of the algorithms was a challenging task, but I was able to overcome it by using the Chrono library in C++. \cite{Measure_time_in_C++}
    \item Lastly, my English proficiency isn’t quite up to par yet, so there’s a possibility that this report contains some grammatical errors.
\end{itemize}


\bibliographystyle{plain}
\bibliography{content/bibl}


\end{document}